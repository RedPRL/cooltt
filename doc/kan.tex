\documentclass[draft]{amsart}
\usepackage{jon-note}
\usepackage{jon-todo}
\usepackage{mathpartir}
\usepackage{mathtools}
\usepackage{varwidth}
\usepackage[landscape,lmargin=.5in,rmargin=.5in]{geometry}

\addbibresource{refs.bib}


\mprset{flushleft}

% misc
\NewDocumentCommand\Hint{m}{{\color{gray}#1}}
\NewDocumentCommand\Code{m}{\mindelim{1}\hat{#1}}
\NewDocumentCommand\Dummy{}{\underline{\hspace{.5em}}}

% cof and [cof]
\NewDocumentCommand\DIM{}{\vvmathbb{I}}
\NewDocumentCommand\COF{}{\vvmathbb{F}}
\NewDocumentCommand\TpPrf{m}{\brk{#1}}
\NewDocumentCommand\Boundary{m}{\partial{#1}}

% types, universe(s), and El
\NewDocumentCommand\IsTp{m}{#1\ \mathit{type}}
\NewDocumentCommand\TpEl{g}{\mathsf{el}\IfValueT{#1}{\prn{#1}}}
\NewDocumentCommand\TpUniv{}{\mathsf{Univ}}
\NewDocumentCommand\CodeUniv{}{\mindelim{1}\widehat{\mathsf{Univ}}}
\NewDocumentCommand\ElInOut{}{{\updownarrow_{\mathsf{el}}}}
\NewDocumentCommand\ElIn{g}{{\uparrow_{\mathsf{el}}}\IfValueT{#1}{\prn{#1}}}
\NewDocumentCommand\ElOut{g}{{\downarrow_{\mathsf{el}}}\IfValueT{#1}{\prn{#1}}}

% Kan operators
\NewDocumentCommand\HCom{mmmmm}{
  \mathsf{hcom}_{#1}\brc{#2\rightsquigarrow #3;#4}\prn{#5}
}
\NewDocumentCommand\Coe{mmmm}{
  \mathsf{coe}_{#1}\brc{#2\rightsquigarrow #3}\prn{#4}
}
\NewDocumentCommand\Com{mmmmm}{
  \mathsf{com}_{#1}\brc{#2\rightsquigarrow #3;#4}\prn{#5}
}

% functions
\NewDocumentCommand\CodeTo{}{
  \mathbin{\hat{\to}}
}
\NewDocumentCommand\CodeProd{mg}{
  {\widehat{\textstyle\prod}}_{#1}\IfValueT{#2}{#2}
}
\NewDocumentCommand\Prod{mg}{
  {\textstyle\prod}_{#1}\IfValueT{#2}{#2}
}
\NewDocumentCommand\TpSub{mmm}{
  \brk{#1 \mathrel{\delimsep{\vert}} #2 \to #3}
}

% pairs
\NewDocumentCommand\CodeSum{mg}{
  {\widehat{\textstyle\sum}}_{#1}\IfValueT{#2}{#2}
}
\NewDocumentCommand\Sum{mg}{
  {\textstyle\sum}_{#1}\IfValueT{#2}{#2}
}
\NewDocumentCommand\TmPair{mm}{
  \gls{#1;#2}
}
\NewDocumentCommand\TmFst{g}{
  \mathsf{fst}\IfValueT{#1}{\prn{#1}}
}
\NewDocumentCommand\TmSnd{g}{
  \mathsf{snd}\IfValueT{#1}{\prn{#1}}
}

% cubical subtypes (not used now)
\NewDocumentCommand\SubIn{m}{
  {\uparrow_{\mathsf{sub}}}\prn{#1}
}
\NewDocumentCommand\SubOut{m}{
  {\downarrow_{\mathsf{sub}}}\prn{#1}
}

% paths (elaborated into functions + cubical subtypes)
\NewDocumentCommand\CodePath{mmm}{
  {\widehat{\mathsf{Path}}}_{#1}\prn{#2;#3}
}
\NewDocumentCommand\TpPath{mmm}{
  {\mathsf{Path}}_{#1}\prn{#2;#3}
}

% abbreviations
\NewDocumentCommand\CodeIsContr{g}{
  \widehat{\mathsf{IsContr}}\IfValueT{#1}{\prn{#1}}
}
\NewDocumentCommand\CodeFiber{gggg}{
  \widehat{\mathsf{Fiber}}\IfValueT{#1}{\prn{#1;#2;#3;#4}}
}
\NewDocumentCommand\CodeEquiv{gg}{
  \widehat{\mathsf{Equiv}}\IfValueT{#1}{\prn{#1;#2}}
}
\NewDocumentCommand\TmEquivApp{gggg}{
  \mathsf{EquivApp}\IfValueT{#1}{\prn{#1;#2;#3\IfValueT{#4}{;#4}}}
}

% V types
\NewDocumentCommand\CodeV{mmmm}{
  \widehat{\mathsf{V}}\brc{#1}\prn{#2;#3;#4}
}
\NewDocumentCommand\TmVin{mmmm}{
  \mathsf{Vin}\brc{#1}\prn{#2;#3;#4}
}
\NewDocumentCommand\TmVproj{mmm}{
  \mathsf{Vproj}\brc{#1}\prn{#2;#3}
}

% composite types
\NewDocumentCommand\TpHCom{mmmm}{
  \mathsf{Hcom}\brc{#1\rightsquigarrow #2;#3}\prn{#4}
}
\NewDocumentCommand\CodeHCom{mmmm}{
  \widehat{\mathsf{Hcom}}\brc{#1\rightsquigarrow #2;#3}\prn{#4}
}
\NewDocumentCommand\TmBox{mmmmm}{
  \mathsf{box}\brc{#1\rightsquigarrow #2;#3}\prn{#4;#5}
}
\NewDocumentCommand\TmCap{mmmmm}{
  \mathsf{cap}\brc{#1\leftsquigarrow #2;#3}\prn{#4;#5}
}

\title{Universes}
\author{St.~Facaan}

\begin{document}
\maketitle

This note gives a semantic version of the structure and algorithms of Kan
universes of Kan types, previously encoded in syntactic form by
\citet{angiuli:2019}. By \emph{semantic} we mean using ordinary families and
functions (as in \citep{orton-pitts:2016,abcfhl:2019}) instead of syntactic
substitutions. We differ from \citep{orton-pitts:2016,abcfhl:2019} in the
following ways:

\begin{enumerate}

  \item The developments in the style of Orton--Pitts construct various notions
    in the internal language of a topos, but do not carefully check that they
    may be combined in a coherent way. In contrast, we are defining an
    equational theory for a \emph{general} coercion and composition operator,
    which can be \emph{modeled} by a carefully executed Orton--Pitts
    construction.

  \item The artificial distinction between ``cap'' and ``tube'' maintained in
    the Orton--Pitts developments is replaced by the more semantic use of
    disjunction. This significantly simplifies both the rules for typing
    compositions, and for computing them.

  \item Finally, we are experimentally targetting \emph{weak} universes \`a la
    Tarski instead of strict ones (i.e.\ we do not place equations on the
    $\TpEl{-}$ operator except those induced by cubical boundary conditions);
    this is both more natural from a semantic point of view, and apparently
    advantageous from the side of implementation (where the strict equations
    lose too much information too early).

\end{enumerate}

\begin{remark}
  The artificial separation of ``cap'' and ``tube'' and the lack of elaboration support of cubical subtypes led to the unfortunate discrepency between the types of arguments of \textsf{hfill} and those of \textsf{hcomp} in Cubical Agda. The main design challenge is that the coherence between ``cap'' and ``tube'' should be mentioned somewhere. One can use cubical subtypes to impose the coherence condition, but cubical subtypes are clumsy without a good elaborator. As a result, Agda removes cubical subtypes from the type of the primitive operator \textsf{hcomp}, using a special checker for the coherence condition beyond its typing. However, the derived function \textsf{hfill} could not enjoy the special treatment of \textsf{hcomp}, thus needing cubical subtypes. In other words, the operators are either semantically correct but clumsy (\textsf{hfill}) or convenient but semantically broken (\textsf{hcomp}). There are at least two ways to fix this:
  \begin{enumerate}
    \item Consolidate ``cap'' and ``tube'' into one system.
    \item Good elaboration support for cubical subtypes.
  \end{enumerate}
  We did both in \texttt{cooltt}.
\end{remark}

\section{Basic rules for the universe}

We begin by a basic weak universe \`a la Tarski closed under dependent product
and sum and, for the sake of simplicity, \emph{itself}; this inconsistent
universe would be replaced by a hierarchy of universes.

\begin{mathparpagebreakable}
  \inferrule[univ formation]{
  }{
    \IsTp{\TpUniv}
  }
  \and
  \inferrule[el formation]{
    \Code{A} : \TpUniv
  }{
    \IsTp{\TpEl{\Code{A}}}
  }
  \and
  \inferrule[pi code]{
    \Code{A} : \TpUniv\\
    \Code{B} : \TpEl{\Code{A}}\to {\TpUniv}
  }{
    \CodeProd{x:\Code{A}}{\Code{B}\prn{x}} : \TpUniv
  }
  \and
  \inferrule[sigma code]{
    \Code{A} : \TpUniv\\
    \Code{B} : \TpEl{\Code{A}}\to {\TpUniv}
  }{
    \CodeSum{x:\Code{A}}{\Code{B}\prn{x}} : \TpUniv
  }
  \and
  \inferrule[universe code]{
  }{
    \CodeUniv : \TpUniv
  }
  \and
  \inferrule[pi decoding]{
    \Code{A} : \TpUniv\\
    \Code{B} : \TpEl{\Code{A}}\to {\TpUniv}
  }{
    \ElInOut : \TpEl{\CodeProd{x:\Code{A}{\Code{B}\prn{x}}}} \cong
    \Prod{x : \TpEl{\Code{A}}}{\TpEl{\Code{B}\prn{x}}}
  }
  \and
  \inferrule[sigma decoding]{
    \Code{A} : \TpUniv\\
    \Code{B} : \TpEl{\Code{A}}\to {\TpUniv}
  }{
    \ElInOut : \TpEl{\CodeSum{x:\Code{A}{\Code{B}\prn{x}}}} \cong
    \Sum{x : \TpEl{\Code{A}}}{\TpEl{\Code{B}\prn{x}}}
  }
  \and
  \inferrule[universe code decoding]{
  }{
    \ElInOut : \TpEl{\CodeUniv} \cong \TpUniv
  }
\end{mathparpagebreakable}

Kan operations are a structure on codes of types.

\begin{mathparpagebreakable}
  \inferrule[hcom structure]{
    \Code{A} : \TpUniv\\
    r,s : \DIM\\
    \phi : \COF\\
    M : \Prod{i:\DIM}{\TpPrf{i=r\lor\phi}\to\TpEl{\Code{A}}}
  }{
    \HCom{\Code{A}}{r}{s}{\phi}{M} : \TpEl{\Code{A}}
    \mid
    r=s\lor\phi \to M\prn{s,*}
  }
  \and
  \inferrule[coe structure]{
    \Code{A} : \DIM\to \TpUniv\\
    r,s:\DIM\\
    M : \TpEl{\Code{A}\prn{r}}
  }{
    \Coe{\Code{A}}{r}{s}{M} : \TpEl{\Code{A}\prn{s}} \mid
    r=s\to M
  }
  \and
  \inferrule[com structure]{
    \Code{A} : \DIM\to \TpUniv\\
    r,s : \DIM\\
    \phi : \COF\\
    M : \Prod{i:\DIM}{\TpPrf{i=r\lor\phi}\to\TpEl{\Code{A}\prn{i}}}
  }{
    \Com{\Code{A}}{r}{s}{\phi}{M} : \TpEl{\Code{A}\prn{s}}
    \mid
    r=s\lor\phi \to M\prn{s,*}
  }
\end{mathparpagebreakable}

In the above, $\mathsf{com}$ is definitionally equal to its standard
decomposition into $\mathsf{hcom}$ and $\mathsf{coe}$.

\subsection{Rules for composite types}

The homogeneous composition structure on the universe corresponds to a code for
composite types; we characterize the elements of this code \emph{directly}
rather than via some isomorphism (this is simplest in the case of cubically
unstable type codes).

\begin{mathparpagebreakable}
  \inferrule[formal hcom code]{
    r,s : \DIM\\
    \phi : \COF\\
    \Code{A} : \Prod{i:\DIM}{\TpPrf{i=r\lor\phi}\to\TpUniv}
  }{
    \CodeHCom{r}{s}{\phi}{\Code{A}} : \TpUniv
    \mid
    r=s\lor\phi
    \to
    \Code{A}\prn{s,*}
  }
  \and
  \inferrule[hcom-univ computation]{
    \Hint{\Code{A} : \Prod{i:\DIM}{\TpPrf{i=r\lor\phi}\to\TpEl{\CodeUniv}}}
  }{
    \HCom{\CodeUniv}{r}{s}{\phi}{\Code{A}} =
    \ElIn{
      \CodeHCom{r}{s}{\phi}{
        \lambda i,*. \ElOut{\Code{A}\prn{i,*}}
      }
    }
    :
    \TpEl{\CodeUniv}
  }
  \and
  \inferrule[composite type introduction]{
    \Hint{r,s:\DIM}\\
    \Hint{\phi:\COF}\\
    \Hint{
      \Code{A} : \Prod{i:\DIM}{
        \TpPrf{i=r\lor\phi}\to\TpUniv
      }
    }\\
    N : \Prod{*:\TpPrf{\phi}}{\TpEl{\Code{A}\prn{s,*}}}\\
    M :
    \TpEl{\Code{A}\prn{r,*}} \mid
    \phi \to
    \Coe{\lambda i.\Code{A}\prn{i,*}}{s}{r}{
      N\prn{*}
    }
  }{
    \TmBox{r}{s}{\phi}{N}{M} : \TpEl{\CodeHCom{r}{s}{\phi}{\Code{A}}}
    \mid
    r=s \to M\mid
    \phi \to N\prn{*}
  }
  \and
  \inferrule[composite type elimination]{
    \Hint{r,s:\DIM}\\
    \Hint{\phi:\COF}\\
    \Code{A} : \Prod{i:\DIM}{
      \TpPrf{i=r\lor\phi}\to\TpUniv
    }
    \\
    M : \TpEl{\CodeHCom{r}{s}{\phi}{\Code{A}}}
  }{
    \TmCap{r}{s}{\phi}{\Code{A}}{M} :
    \TpEl{\Code{A}\prn{r,*}}
    \mid
    r=s\to M
    \mid
    \phi\to\Coe{\lambda i.\Code{A}\prn{i,*}}{s}{r}{M}
  }
  \and
  \inferrule[composite type computation]{
    \Hint{r,s:\DIM}\\
    \Hint{\phi:\COF}\\
    \Hint{
      \Code{A} : \Prod{i:\DIM}{
        \TpPrf{i=r\lor\phi}\to\TpUniv
      }
    }\\
    \Hint{N : \Prod{*:\TpPrf{\phi}}{\TpEl{\Code{A}\prn{s,*}}}}\\
    \Hint{
      M :
      \TpEl{\Code{A}\prn{r,*}}
      \mid
      \phi\to
      \Coe{\lambda i.\Code{A}\prn{i,*}}{s}{r}{
        N\prn{*}
      }
    }
  }{
    \TmCap{r}{s}{\phi}{\Code{A}}{
      \TmBox{r}{s}{\phi}{N}{M}
    }
    =
    M
    : \TpEl{\Code{A}\prn{r,*}}
  }
  \and
  \inferrule[composite type uniqueness]{
    \Hint{r,s:\DIM}\\
    \Hint{\phi:\COF}\\
    \Hint{
      \Code{A} : \Prod{i:\DIM}{
        \TpPrf{i=r\lor\phi}\to\TpUniv
      }
    }\\
    \Hint{
      M : \TpEl{\CodeHCom{r}{s}{\phi}{\Code{A}}}
    }
  }{
    M =
    \TmBox{r}{s}{\phi}{
      \lambda{*}. M
    }{
      \TmCap{r}{s}{\phi}{\Code{A}}{M}
    }
    : \TpEl{\CodeHCom{r}{s}{\phi}{\Code{A}}}
  }
\end{mathparpagebreakable}

\begin{warning}
  %
  The arguments to $\mathsf{box}$ go in the order forced by their typing
  constraints, but please note that this is the opposite of what appeared in
  \citep{angiuli:2019}.
  %
\end{warning}



\subsection{Rules for V types}

We begin by defining some definitional extensions for equivalences.
\[
  \TpPath{A}{M}{N} \coloneq \Prod{i:\DIM}{A\prn{i} \mid i=0 \to M \mid i=1 \to N}
\]
\[
  \begin{array}{l}
    \CodeIsContr : \TpUniv\to\TpUniv\\
    \CodeIsContr{\Code{C}} \coloneq \CodeSum{x:\Code{C}}{\CodeProd{y:\Code{C}}{\CodePath{\lambda\Dummy.\Code{C}}{x}{y}}}
  \end{array}
\]
\[
  \begin{array}{l}
    \CodeFiber : \Prod{\Code{A}:\TpUniv}{\Prod{\Code{B}:\TpUniv}{\prn{\TpEl{\Code{A}}\to\TpEl{\Code{B}}}\to\TpEl{\Code{B}}\to\TpUniv}}\\
    \CodeFiber{\Code{A}}{\Code{B}}{f}{y} \coloneq \CodeProd{x:\Code{A}}{\CodePath{\lambda\Dummy.\Code{B}}{f\prn{x}}{y}}
  \end{array}
\]
\[
  \begin{array}{l}
    \CodeEquiv : \TpUniv\to\TpUniv\to\TpUniv\\
    \CodeEquiv{\Code{A}}{\Code{B}} \coloneq
    \CodeSum{
      f:\Code{A}\CodeTo\Code{B}
    }{
      \CodeProd{y:\Code{B}}{
        \CodeIsContr{\CodeFiber{\Code{A}}{\Code{B}}{\ElOut{f}}{y}}
      }
    }
  \end{array}
\]
\[
  \begin{array}{l}
    \TmEquivApp : \Prod{\Code{A}:\TpUniv}{\Prod{\Code{B}:\TpUniv}{\TpEl{\CodeEquiv{\Code{A}}{\Code{B}}}\to\TpEl{\Code{A}}\to\TpEl{\Code{B}}}}\\
    \TmEquivApp{\Code{A}}{\Code{B}}{E} \coloneq \ElOut{\TmFst{\ElOut{E}}}
  \end{array}
\]

\begin{warning}
  Contractability is defined in terms of paths \emph{to} the center, not \emph{from} it.
\end{warning}

We may now specify the rules for the V types.
\begin{mathparpagebreakable}
  \inferrule[v type code]{
    r : \DIM\\
    \Code{A} : \TpPrf{r=0}\to\TpUniv\\
    \Code{B} : \TpUniv\\
    E : \Prod{*:\TpPrf{r=0}}{\TpEl{\CodeEquiv{\Code{A}\prn{*}}{\Code{B}}}}
  }{
    \CodeV{r}{\Code{A}}{\Code{B}}{E} : \TpUniv
    \mid
    r=0
    \to
    \Code{A}\prn{*}
    \mid
    r=1
    \to
    \Code{B}
  }
\end{mathparpagebreakable}


\hfill

\begin{mathparpagebreakable}
  \inferrule[v type introduction]{
    \Hint{r:\DIM}\\
    \Hint{\Code{A} : \TpPrf{r=0}\to\TpUniv}\\
    \Hint{\Code{B} : \TpUniv}\\
    \Hint{E : \Prod{*:\TpPrf{r=0}}{\TpEl{\CodeEquiv{\Code{A}\prn{*}}{\Code{B}}}}}\\
    \\\\
    M : \Prod{*:\TpPrf{r=0}}{\TpEl{\Code{A}\prn{*}}}\\
    N : \TpEl{\Code{B}} \mid r=0 \to \TmEquivApp{\Code{A}\prn{*}}{\Code{B}}{E\prn{*}}{M\prn{*}}
  }{
    \TmVin{r}{E}{M}{N} : \TpEl{\CodeV{r}{\Code{A}}{\Code{B}}{E}}
    \mid r=0 \to M
    \mid r=1 \to N
  }
  \and
  \inferrule[v type elimination]{
    \Hint{r:\DIM}\\
    \Hint{\Code{A} : \TpPrf{r=0}\to\TpUniv}\\
    \Hint{\Code{B} : \TpUniv}\\
    \Hint{E : \Prod{*:\TpPrf{r=0}}{\TpEl{\CodeEquiv{\Code{A}\prn{*}}{\Code{B}}}}}\\
    M : \TpEl{\CodeV{r}{\Code{A}}{\Code{B}}{E}}
  }{
    \TmVproj{r}{E}{M} : \TpEl{\Code{B}}
    \mid r=0 \to \TmEquivApp{\Code{A}\prn{*}}{\Code{B}}{E\prn{*}}{M}
    \mid r=1 \to M
  }
  \and
  \inferrule[v type computation]{
    \Hint{r:\DIM}\\
    \Hint{\Code{A} : \TpPrf{r=0}\to\TpUniv}\\
    \Hint{\Code{B} : \TpUniv}\\
    \Hint{E : \Prod{*:\TpPrf{r=0}}{\TpEl{\CodeEquiv{\Code{A}\prn{*}}{\Code{B}}}}}\\
    \\\\
    \Hint{M : \Prod{*:\TpPrf{r=0}}{\TpEl{\Code{A}\prn{*}}}}\\
    \Hint{N : \TpEl{\Code{B}} \mid r=0 \to \TmEquivApp{\Code{A}\prn{*}}{\Code{B}}{E\prn{*}}{M\prn{*}}}
  }{
    \TmVproj{r}{E}{\TmVin{r}{E}{M}{N}} = N : \TpEl{\Code{B}}
  }
  \and
  \inferrule[v type uniqueness]{
    \Hint{r:\DIM}\\
    \Hint{\Code{A} : \TpPrf{r=0}\to\TpUniv}\\
    \Hint{\Code{B} : \TpUniv}\\
    \Hint{E : \Prod{*:\TpPrf{r=0}}{\TpEl{\CodeEquiv{\Code{A}\prn{*}}{\Code{B}}}}}\\
    \Hint{M : \TpEl{\CodeV{r}{\Code{A}}{\Code{B}}{E}}}
  }{
    M = \TmVin{r}{E}{\lambda{*}.M}{\TmVproj{r}{E}{M}} : \TpEl{\CodeV{r}{\Code{A}}{\Code{B}}{E}}
  }
  \and
\end{mathparpagebreakable}


\section{Composition and coercion algorithms}

\[
  \inferrule{
    \Hint{s:\DIM}\\
    \Hint{\Code{A} : \TpPrf{s=0}\to\TpUniv}\\
    \Hint{\Code{B} : \TpUniv}\\
    \Hint{E : \Prod{*:\TpPrf{s=0}}{\TpEl{\CodeEquiv{\Code{A}\prn{*}}{\Code{B}}}}}\\
    \\\\
    \Hint{r,r':\DIM}\\
    \Hint{
      M : \Prod{i:\DIM}{
        \TpPrf{i=r\lor\phi}\to\TpEl{
          \CodeV{s}{\Code{A}}{\Code{B}}{E}
        }
      }
    }
    \\\\\\
    {
      \begin{array}{l}
        \tilde{O} : \Prod{\Code{X}:\TpUniv}{\prn{\Prod{i:\DIM}{\TpPrf{i=r\lor\phi}\to\TpEl{\Code{X}}}}\to\DIM\to\TpEl{X}}\\
        \tilde{O}\prn{\Code{X},N,i} \coloneq \HCom{\Code{X}}{r}{i}{\phi}{N}
      \end{array}
    }
    \\\\\\
    {
      \begin{array}{l}
        \tilde{P} : \Prod{i:\DIM}{\TpPrf{i=r\lor\phi\lor\Boundary{s}}\to\TpEl{\Code{B}}}\\
        \tilde{P}\prn{i,*} \coloneq \brk{
            i=r\lor\phi \to \TmVproj{s}{E}{M\prn{i,*}} \mid
            s=0 \to \TmEquivApp{\Code{A}\prn{*}}{\Code{B}}{E\prn{*}}{O\prn{\Code{A}\prn{*},M,i}} \mid
            s=1 \to O\prn{\Code{B},M,i}
        }
      \end{array}
    }
  }{
    \HCom{
      \CodeV{s}{\Code{A}}{\Code{B}}{E}
    }{r}{r'}{\phi}{M}
    =
    \TmVin{s}{E}{\lambda{*}.\tilde{O}\prn{\Code{A}\prn{*},M,r'}}{\HCom{\Code{B}}{r}{r'}{\phi\lor\Boundary{s}}{\tilde{P}}}
    :
    \TpEl{
      \CodeV{s}{\Code{A}}{\Code{B}}{E}
    }
  }
\]

\[
  \inferrule{
    \Hint{s_\bullet : \DIM\to\DIM}\\
    \Hint{\Code{A}_\bullet : \Prod{i:\DIM}{\TpPrf{s_i=0}\to\TpUniv}}\\
    \Hint{\Code{B}_\bullet : \DIM\to\TpUniv}\\
    \Hint{E_\bullet : \Prod{i:\DIM}{\Prod{*:\TpPrf{s_i=0}}{\TpEl{\CodeEquiv{\Code{A}\prn{*}}{\Code{B}}}}}}\\
    \\\\
    \Hint{r,r':\DIM}\\
    \Hint{M : \TpEl{\CodeV{s_r}{\Code{A}_r}{\Code{B}_r}{E_r}}}
    \\\\\\
    {
      \begin{array}{l}
        \tilde{F} : \Prod{i:\DIM}{\TpPrf{s_i=0}\to\TpEl{\Code{A}_i\prn{*}}\to\TpEl{\Code{B}_i}}
        \\
        \tilde{F}\prn{i,*} \coloneq \TmEquivApp{\Code{A}_{i}\prn{*}}{\Code{B}_{i}}{E_{i}\prn{*}}
      \end{array}
    }
    \\\\\\
    {
      \begin{array}{l}
        \tilde{O} : \TpEl{\Code{B}_{r'}}
        \\
        \tilde{O} \coloneq \Coe{\Code{B}_\bullet}{r}{r'}{\TmVproj{s_r}{E_r}{M}}
      \end{array}
    }
    \\\\\\
    {
      \begin{varwidth}{\textwidth}
        \[
          \begin{array}{l}
            \tilde{P} : \TpPrf{s_{r'}=0}\to
            \Prod{x:\TpEl{\Code{A}_{r'}\prn{*}}}{
              \TpPath{\lambda\Dummy.\TpEl{\Code{B}_{r'}}}{\tilde{F}\prn{r',*,x}}{\tilde{O}}
              \to
              \DIM\to\TpEl{\CodeFiber{\Code{A}_{r'}\prn{*}}{\Code{B}_{r'}}{\tilde{F}\prn{r',*}}{\tilde{O}}}
            }
            \\
            \tilde{P}\prn{*,M,N} \coloneq
            \ElOut{
              \ElOut{
                \TmSnd{\ElOut{
                    \ElOut{
                      \TmSnd{\ElOut{
                          E_{r'}\prn{*}
                      }}
                    }\prn{\tilde{O}}
                }}
              }\prn{\ElIn{\TmPair{M}{\ElIn{N}}}}
            }
          \end{array}
        \]
      \end{varwidth}
    }
    \\\\\\
    { % Favonia: this is just coe in paths, but this style leads to shorter code?
      \begin{array}{l}
        \tilde{Q} :
        \TpPrf{\forall k.s_k=0}\to
        \TpPath{\lambda\Dummy.\TpEl{\Code{B}_{r'}}}{
          \tilde{F}\prn{r',*,\Coe{\Code{A}_\bullet\prn{*}}{r}{r'}{M}}
        }{
          \tilde{O}
        }
        %\\
        %\tilde{Q}\prn{*,j} \coloneq
        %\Com{\Code{B}_\bullet}{r}{r'}{\Boundary{j}}{
        %  \lambda i,*.
        %  \brk{
        %    i=r\lor j=0 \to \tilde{F}\prn{i,*,\Coe{\Code{A}_\bullet\prn{*}}{r}{i}{M}}
        %    \mid
        %    j=1\to \Coe{\Code{B}_\bullet}{r}{i}{\tilde{F}\prn{r,*,M}}
        %  }
        %}
        \\
        \tilde{Q}\prn{*} \coloneq
        \ElOut{\Coe{
          \lambda i.
          \CodePath{
            \lambda\Dummy.\TpEl{\Code{B}_i}
          }{
            \tilde{F}\prn{i,*,\Coe{\Code{A}_\bullet\prn{*}}{r}{i}{M}}
          }{
            \Coe{\Code{B}_\bullet}{r}{i}{\tilde{F}\prn{r,*,M}}
          }
        }{r}{r'}{
          \ElIn{\lambda\Dummy.\tilde{F}\prn{r,*,M}}
        }}
      \end{array}
    }
    \\\\\\
    {\begin{varwidth}{\textwidth}
    \[
      \begin{array}{l}
        \tilde{R} : \TpPrf{s_{r'}=0}\to\CodeFiber{\Code{A}_{r'}\prn{*}}{\Code{B}_{r'}}{E_{r'}}{\tilde{O}}
        \\
        \tilde{R}\prn{*} \coloneq
          \HCom{\CodeFiber{\Code{A}_{r'}\prn{*}}{\Code{B}_{r'}}{E_{r'}}{\tilde{O}}}{1}{0}{\parens{\forall k.s_k = 0}\lor r=r'}{
          \lambda j,*.
          \brk{ %CJHM
            \begin{aligned}
              &\prn{\forall k.s_k = 0}\to\tilde{P}\prn{*,\Coe{\Code{A}_\bullet\prn{*}}{r}{r'}{M},\tilde{Q}\prn{*},j}
              \\
              &r=r' \to \tilde{P}\prn{*,M,\prn{\lambda\Dummy.\TmVproj{s_r}{E_r}{M}},j}
              \\
              &j=1 \to
                \TmFst{\ElOut{
                  \ElOut{
                    \TmSnd{\ElOut{
                        E_{r'}\prn{*}
                    }}
                  }\prn{\tilde{O}}
                }}
            \end{aligned}
          }
        }
      \end{array}
    \]
    \end{varwidth}
    }
    \\\\\\
    {
      \begin{array}{l}
        \tilde{S} : \TpEl{\Code{B}_{r'}}\\
        \tilde{S} \coloneq \HCom{\TpEl{\Code{B}_{r'}}}{1}{0}{r=r'\lor\Boundary{s_{r'}}}{
          \lambda k,*.
          \brk{
            k=1\lor s_{r'}=1\lor r=r'\to \tilde{O}
            \mid
            s_{r'}=0 \to \ElOut{\TmSnd{\ElOut{\tilde{R}\prn{*}}}}\prn{k}
          }
        }
      \end{array}
    }
  }{
    \Coe{
      \CodeV{s_\bullet}{\Code{A}_\bullet}{\Code{B}_\bullet}{E_\bullet}
    }{r}{r'}{\phi}{M}
    =
    \TmVin{s_{r'}}{E_{r'}}{\lambda{*}.\TmFst{\ElOut{\tilde{R}\prn{*}}}}{\tilde{S}}
    :
    \TpEl{\CodeV{s_\bullet}{\Code{A}_{r'}}{\Code{B}_{r'}}{E_{r'}}}
  }
\]

\[
  \inferrule{
    \Hint{s,s' : \DIM}\\
    \Hint{\psi : \COF}\\
    \Hint{\Code{A} : \Prod{j:\DIM}{\TpPrf{j=s\lor\psi}\to\TpUniv}}\\\\
    \Hint{r,r' : \DIM}\\
    \Hint{\phi : \COF}\\
    \Hint{
      M : \Prod{i:\DIM}{
        \Prod{*:\TpPrf{i=r\lor\phi}}{
          \TpEl{
            \CodeHCom{s}{s'}{\psi}{\Code{A}}
          }
        }
      }
    }
    \\\\\\
    {
      \begin{array}{l}
        \tilde{O} : \Prod{i:\DIM}{\Prod{*:\TpPrf{i=r\lor \phi}}{\TpEl{\Code{A}\prn{s,*}}}}\\
        \tilde{O}\prn{i,*} \coloneq \TmCap{s}{s'}{\psi}{\Code{A}}{M\prn{i,*}}
      \end{array}
    }
    \\\\\\
    {
      \begin{array}{l}
        \tilde{P} : \Prod{i:\DIM}{\Prod{*:\TpPrf{s=s'\lor \psi}}{\TpEl{\Code{A}\prn{s',*}}}}\\
        \tilde{P}\prn{i,*} = \HCom{\Code{A}\prn{s',*}}{r}{i}{\phi}{M}
      \end{array}
    }
    \\\\\\
    {
      \begin{array}{l}
        \tilde{T} : \Prod{i:\DIM}{\Prod{*:\TpPrf{i=r\lor\phi\lor\psi\lor{s=s'}}}{\TpEl{\Code{A}\prn{s,*}}}}\\
        \tilde{T}\prn{i,*} \coloneq
        \brk{
          i=r\lor\phi \to \tilde{O}\prn{i,*}
          \mid
          \psi \to \Coe{\lambda j. \Code{A}\prn{j,*}}{s'}{s}{\tilde{P}\prn{i,*}}
          \mid
          s=s' \to \tilde{P}\prn{i,*}
        }
      \end{array}
    }
  }{
    \HCom{
      \CodeHCom{s}{s'}{\psi}{\Code{A}}
    }{r}{r'}{\phi}{M}
    =
    \TmBox{s}{s'}{\psi}{\lambda{*}.\tilde{P}\prn{r',*}}{
      \HCom{\Code{A}\prn{s,*}}{r}{r'}{\phi\lor\psi\lor{s=s'}}{\tilde{T}}
    }
    :
    \TpEl{
      \CodeHCom{s}{s'}{\psi}{\Code{A}}
    }
  }
\]

\[
  \inferrule{
    \Hint{s_\bullet, s'_\bullet : \DIM\to\DIM}\\
    \Hint{\phi_\bullet : \DIM\to\COF}\\
    \Hint{
      \Code{A}_\bullet :
      \Prod{i:\DIM}{
        \Prod{j:\DIM}{
          \Prod{*:\TpPrf{j=s_{i}\lor{\phi_{i}}}}{
            \TpUniv
          }
        }
      }
    }
    \\\\
    \Hint{r,r' : \DIM}\\
    \Hint{
      M : \TpEl{
        \CodeHCom{s_{r}}{s'_{r}}{\phi_{r}}{\Code{A}_{r}}
      }
    }
    \\\\\\
    {
      \begin{array}{l}
        \tilde{N} : \Prod{i,j:\DIM}{
          \Prod{*:\TpPrf{\forall i.\phi_i}}{
            \TpEl{\Code{A}_{i}\prn{j,*}}
          }
        }\\
        \tilde{N}\prn{i,j,*} \coloneq
        \Coe{\Code{A}_i\prn{\bullet,*}}{s'_i}{j}{\Coe{\Code{A}_\bullet\prn{s'_\bullet,*}}{r}{i}{M}}
      \end{array}
    }
    \\\\\\
    {
      \begin{array}{l}
        \tilde{O} : \DIM \to \TpEl{\Code{A}_{r}\prn{s_r,*}}
        \\
        \tilde{O}\prn{j} \coloneq
        \HCom{\Code{A}_{r}\prn{s_r,*}}{s'_{r}}{j}{\phi_{r}}{
          \lambda k,*.
          \brk{
            k=s'_r \to \TmCap{s_r}{s'_r}{\phi_r}{\Code{A}_r}{M}
            \mid
            \phi_r \to
            \Coe{\Code{A}_r\prn{\bullet,*}}{k}{s_r}{
              \Coe{\Code{A}_r\prn{\bullet,*}}{s'_r}{k}{M}
            }
          }
        }
      \end{array}
    }
    \\\\\\
    {
      \begin{array}{l}
        \tilde{P} : \TpEl{\Code{A}_{r'}\prn{s_{r'},*}} \\
        \tilde{P} \coloneq
        \Com{\Code{A}_\bullet\prn{s_\bullet,*}}{r}{r'}{\forall i.\phi_i \lor \forall i.\prn{s_i=s'_i}}{
          \lambda i,*.
          \brk{
            i=r \to \tilde{O}\prn{s_r}
            \mid
            \forall i.\phi_i \to \tilde{N}\prn{i,s_i,*}
            \mid
            \forall i.(s_i=s'_i) \to \Coe{\Code{A}_\bullet\prn{s_\bullet,*}}{r}{i}{M}
          }
        }
      \end{array}
    }
    \\\\\\
    {
      \begin{array}{l}
        \tilde{Q} : \Prod{j:\DIM}{
          \Prod{*:\TpPrf{\phi_{r'}}}{
            \TpEl{\Code{A}_{r'}\prn{j,*}}
          }
        }\\
        \tilde{Q}\prn{j,*} \coloneq
        \Com{\Code{A}_{r'}\prn{\bullet,*}}{s_{r'}}{j}{r=r'\lor\forall i.\phi_i}{
          \lambda k,*.\brk{
            k=s_{r'} \to \tilde{P}
            \mid
            r=r' \to \Coe{\Code{A}_{r'}\prn{\bullet,*}}{s'_{r'}}{k}{M}
            \mid
            \forall i.\phi_i \to \tilde{N}\prn{r',k,*}
          }
        }
      \end{array}
    }
    \\\\\\
    {
      \begin{array}{l}
        \tilde{H} : \TpEl{\Code{A}_{r'}\prn{s_{r'},*}} \\
        \tilde{H} \coloneq
        \HCom{\Code{A}_{r'}(s_{r'},*)}{s_{r'}}{s'_{r'}}{\phi_{r'} \lor r=r'}{
          \lambda j,*.\brk{
            j=s_{r'}\to \tilde{P}
            \mid
            \phi_{r'} \to \Coe{\Code{A}_{r'}\prn{\bullet,*}}{j}{s_{r'}}{\tilde{Q}\prn{j,*}}
            \mid
            r=r' \to \tilde{O}\prn{j}
          }
        }
      \end{array}
    }
  }{
    \Coe{
      \lambda i.
      \CodeHCom{s_{i}}{s'_{i}}{\phi_{i}}{\Code{A}_i}
    }{r}{r'}{M}
    =
    \TmBox{s_{r'}}{s'_{r'}}{\phi_{r'}}{
      \lambda{*}.{\tilde{Q}\prn{s'_{r'},*}}
    }{
      \tilde{H}
    }
    :
    \TpEl{
      \CodeHCom{s_{r'}}{s'_{r'}}{\phi_{r'}}{\Code{A}_{r'}}
    }
  }
\]

\clearpage
\nocite{*}
\printbibliography

\end{document}
