\documentclass[final]{amsart}
\usepackage{jon-note}
\usepackage{jon-todo}
\usepackage{mathpartir}
\usepackage{mathtools}
\usepackage{pdflscape}

\addbibresource{refs.bib}


\mprset{flushleft}
\NewDocumentCommand\redtt{}{%
  \texttt{\textcolor[rgb]{.91,.31,.27}{red}tt}%
}
\NewDocumentCommand\RedPRL{}{%
  \textsf{\textcolor[rgb]{.91,.31,.27}{Red}PRL}%
}


\NewDocumentCommand\delimmid{}{\mathrel{\delimsep{\vert}}}

% misc
\NewDocumentCommand\Hint{m}{{\color{gray}#1}}
\NewDocumentCommand\Code{m}{\mindelim{1}\hat{#1}}
\NewDocumentCommand\Dummy{}{\underline{\hspace{.5em}}}

% cof and [cof]
\NewDocumentCommand\DIM{}{\vvmathbb{I}}
\NewDocumentCommand\COF{}{\vvmathbb{F}}
\NewDocumentCommand\TpPrf{m}{\brk{#1}}
\NewDocumentCommand\Boundary{m}{\partial{#1}}
\NewDocumentCommand\IsTrue{m}{#1\ \mathit{true}}

% types, universe(s), and El
\NewDocumentCommand\IsTp{m}{#1\ \mathit{type}}
\NewDocumentCommand\TpEl{g}{\mathsf{el}\IfValueT{#1}{\prn{#1}}}
\NewDocumentCommand\TpUniv{}{\mathsf{Univ}}
\NewDocumentCommand\CodeUniv{}{\mindelim{1}\widehat{\mathsf{Univ}}}
\NewDocumentCommand\ElInOut{}{{\updownarrow_{\mathsf{el}}}}
\NewDocumentCommand\ElIn{sg}{{\uparrow_{\mathsf{el}}}\IfValueT{#2}{\IfBooleanTF{#1}{#2}{\prn{#2}}}}
\NewDocumentCommand\ElOut{sg}{{\downarrow_{\mathsf{el}}}\IfValueT{#2}{\IfBooleanTF{#1}{#2}{\prn{#2}}}}

% Kan operators
\NewDocumentCommand\HCom{mmmmm}{
  \mathsf{hcom}_{#1}\brc{#2\rightsquigarrow #3;#4}\prn{#5}
}
\NewDocumentCommand\Coe{mmmm}{
  \mathsf{coe}_{#1}\brc{#2\rightsquigarrow #3}\prn{#4}
}
\NewDocumentCommand\Com{mmmmm}{
  \mathsf{com}_{#1}\brc{#2\rightsquigarrow #3;#4}\prn{#5}
}

% functions
\NewDocumentCommand\CodeTo{}{
  \mathbin{\hat{\to}}
}
\NewDocumentCommand\CodeProd{mg}{
  {\widehat{\textstyle\prod}}_{#1}\IfValueT{#2}{#2}
}
\NewDocumentCommand\Prod{mg}{
  {\textstyle\prod}_{#1}\IfValueT{#2}{#2}
}
\NewDocumentCommand\TpSub{mmm}{
  \mathsf{Sub}\prn{#1; #2; #3}
}

% pairs
\NewDocumentCommand\CodeSum{mg}{
  {\widehat{\textstyle\sum}}_{#1}\IfValueT{#2}{#2}
}
\NewDocumentCommand\Sum{mg}{
  {\textstyle\sum}_{#1}\IfValueT{#2}{#2}
}
\NewDocumentCommand\TmPair{mm}{
  \gls{#1;#2}
}
\NewDocumentCommand\TmFst{g}{
  \mathsf{fst}\IfValueT{#1}{\prn{#1}}
}
\NewDocumentCommand\TmSnd{g}{
  \mathsf{snd}\IfValueT{#1}{\prn{#1}}
}

% cubical subtypes
\NewDocumentCommand\SubIn{m}{
  {\uparrow_{\mathsf{sub}}}\prn{#1}
}
\NewDocumentCommand\SubOut{m}{
  {\downarrow_{\mathsf{sub}}}\prn{#1}
}

% paths
\NewDocumentCommand\CodePath{mmm}{
  \mindelim{1}{\widehat{\mathsf{Path}}}_{#1}\prn{#2;#3}
}
\NewDocumentCommand\TpPath{mmm}{
  \mindelim{1}{\mathsf{Path}}_{#1}\prn{#2;#3}
}

% abbreviations
\NewDocumentCommand\CodeIsContr{g}{
  \widehat{\mathsf{IsContr}}\IfValueT{#1}{\prn{#1}}
}
\NewDocumentCommand\CodeFiber{gggg}{
  \widehat{\mathsf{Fiber}}\IfValueT{#1}{\prn{#1;#2;#3;#4}}
}
\NewDocumentCommand\CodeEquiv{gg}{
  \widehat{\mathsf{Equiv}}\IfValueT{#1}{\prn{#1;#2}}
}
\NewDocumentCommand\TmEquivApp{gggg}{
  \mathsf{EquivApp}\IfValueT{#1}{\prn{#1;#2;#3\IfValueT{#4}{;#4}}}
}

% V types
\NewDocumentCommand\CodeV{mmmm}{
  \widehat{\mathsf{V}}\brc{#1}\prn{#2;#3;#4}
}
\NewDocumentCommand\TmVin{mmmm}{
  \mathsf{Vin}\brc{#1}\prn{#2;#3;#4}
}
\NewDocumentCommand\TmVproj{mmmmm}{
  \mathsf{Vproj}\brc{#1}\prn{#2;#3;#4;#5}
}

% composite types
\NewDocumentCommand\TpHCom{mmmm}{
  \mathsf{Hcom}\brc{#1\rightsquigarrow #2;#3}\prn{#4}
}
\NewDocumentCommand\CodeHCom{mmmm}{
  \widehat{\mathsf{Hcom}}\brc{#1\rightsquigarrow #2;#3}\prn{#4}
}
\NewDocumentCommand\TmBox{mmmmm}{
  \mathsf{box}\brc{#1\rightsquigarrow #2;#3}\prn{#4;#5}
}
\NewDocumentCommand\TmCap{mmmmm}{
  \mathsf{cap}\brc{#1\leftsquigarrow #2;#3}\prn{#4;#5}
}

\newtheorem{cool}[theorem]{Cool Remark}
\SetupFleuron{cool}

\title{A cool Cartesian cubical type theory}
\author{Carlo Angiuli}
\author{Evan Cavallo}
\author{Kuen-Bang Hou (Favonia)}
\author{Jonathan Sterling}

\begin{document}
\maketitle

We present a semantic version of Cartesian cubical type theory, previously
encoded in syntactic form by \citet{angiuli:2019}, corresponding to an idealized
\texttt{cooltt} \citep{cooltt:2020}. By \emph{semantic} we mean that we use
ordinary families and functions (as in \citep{orton-pitts:2016,abcfhl:2019})
instead of syntactic substitutions. We differ from
\citep{orton-pitts:2016,abcfhl:2019} in the following ways:

\begin{enumerate}

  \item The developments in the style of Orton--Pitts construct various notions
    in the internal language of a topos, but do not always carefully check that
    they may be combined in a coherent way, nor whether they are expressible purely in
    the language of finite limits (a pre-requisite for defining a type theory).
    In contrast, we are defining an equational theory for a \emph{general}
    coercion and composition operator, which can be \emph{modeled} by a
    carefully executed Orton--Pitts construction.

  \item The artificial distinction between ``cap'' and ``tube'' maintained in
    the Orton--Pitts developments is replaced by the more semantic use of
    disjunction. This significantly simplifies both the rules for typing
    compositions, and for computing them.

  \item Finally, we are experimentally targeting \emph{weak} universes \`a la
    Tarski instead of strict ones (i.e.\ we do not place equations on the
    $\TpEl{-}$ operator except those induced by cubical boundary conditions);
    this is both more natural from a semantic point of view, and apparently
    advantageous from the side of implementation.
    The strict equations lose too much information too early, such as the Kan-ness
    and universe levels. For example, it is difficult to recognize
    \[
      \Prod{i : \DIM}{
        \TpSub{\TpEl{\Code{A}\prn{i}}}{\Boundary{i}}{
          \brk{i=0 \to M \delimmid i=1 \to N}
        }
      }
    \]
    as a path type ``$\TpPath{A}{M}{N}$'' and infer its Kan operators.
    (It will be next to impossible with extension types or other fancy types.)

\end{enumerate}

\section{Cubical structure}

We leave implicit the basic rules of type theory, including pi and sigma types
equipped with $\eta$ rules. As in \citep{orton-pitts:2016,abcfhl:2019}, we
expose the cubical nature of our setting through $\DIM$, the Yoneda embedding of
the interval, and $\COF$, a universe of strict propositions. (Note that
$\IsTp{A}$ means that $A$ is a \emph{cubical set}, not that $A$ is Kan.)

\begin{mathparpagebreakable}
  \inferrule[$\DIM$-formation]{
  }{
    \IsTp{\DIM}
  }
  \and
  \inferrule[$\DIM$-intro]{
  }{
    0,1 : \DIM
  }
  \and
  \inferrule[$\COF$-formation]{
  }{
    \IsTp{\COF}
  }
  \\
  \inferrule[cofibration intro]{
    r,s : \DIM
  }{
    r = s : \COF
  }
  \and
  \inferrule{
    \phi,\psi : \COF
  }{
    \phi\lor\psi : \COF
  }
  \and
  \inferrule{
    \phi,\psi : \COF
  }{
    \phi\land\psi : \COF
  }
  \and
  \inferrule{
    \phi : \DIM\to\COF
  }{
    \forall i.\phi(i) : \COF
  }
  \\
  \inferrule[prf formation]{
    \phi : \COF
  }{
    \IsTp{\TpPrf{\phi}}
  }
  \and
  \inferrule[prf intro]{
    \Hint{\phi:\COF} \\
    \IsTrue{\phi}
  }{
    * : \TpPrf{\phi}
  }
  \and
  \inferrule[prf elim]{
    \Hint{\phi:\COF} \\
    p : \TpPrf{\phi}
  }{
    \IsTrue{\phi}
  }
  \and
  \inferrule[prf uniqueness]{
    \Hint{\phi : \COF} \\
    p : \TpPrf{\phi}
  }{
    p = * : \TpPrf{\phi}
  }
  \and
  \inferrule[prf extensionality]{
    \Hint{\phi,\psi : \COF} \\
    \IsTrue{\phi}\vdash\IsTrue{\psi}\\
    \IsTrue{\psi}\vdash\IsTrue{\phi}
  }{
    \phi = \psi : \COF
  }
\end{mathparpagebreakable}

Truth of each $\phi:\COF$ has the expected meaning. In particular:
%
\begin{enumerate}
  \item under a hypothesis $\TpPrf{r=s}$, we have $r=s:\DIM$;

  \item $\TpPrf{0=1}$ is the false proposition, whose elimination is the empty
  case split $\brk{}$; and

  \item elimination for a disjunction $*:\TpPrf{\phi\lor\psi}$ is a case split
  $\brk{\phi \to c_\phi \delimmid \psi \to c_\psi}$ whose branches must agree on
  $\phi\land\psi$:

  \begin{mathparpagebreakable}
    \inferrule{
      \Hint{\phi,\psi : \COF} \\
      \IsTrue{\phi\lor\psi} \\
      \IsTp{C} \\
      \IsTrue{\phi} \vdash c_\phi : C \\
      \IsTrue{\psi} \vdash c_\psi : C \\
      \IsTrue{\phi\land\psi} \vdash c_\phi = c_\psi : C
    }{
      \brk{\phi \to c_\phi \delimmid \psi \to c_\psi} : C
    }
  \end{mathparpagebreakable}
\end{enumerate}
%
We extend the case split notation to $n$-ary disjunctions as needed.

\begin{cool}
Although this presentation allows for terms $\DIM\to\COF$, etc., it is important
for our implementation that we disallow $\DIM$ and $\COF$ from appearing ``on
the right'' in connectives. That is, while \texttt{cooltt} users can write
$\DIM\to A$, they cannot write $A\to\DIM$. This ensures that the equational
theory of $\DIM$ and $\COF$ do not depend on the equational theory of general
terms. It also allows us to define $\forall i.\phi(i)$ by cases on $\phi$.
\end{cool}

We will make particular use of the \emph{boundary} cofibration:

\[
  \begin{array}{l}
    \Boundary{\bullet} : \DIM\to\COF\\
    \Boundary{r} \coloneq \prn{r=0 \lor r=1}
  \end{array}
\]

The role of $\COF$ is to allow us to (partially) specify the boundaries of
elements. If $M$ is a partial element of $A$ defined under $\IsTrue{\phi}$, we
write $N : A \mid \phi \to M$ to express that $N$ is a total element of $A$ that
restricts under $\phi$ to $M$:

\begin{mathparpagebreakable}
  \mprset{fraction={===}}
  \inferrule{
    \IsTp{A} \\
    \phi : \COF \\
    \IsTrue{\phi} \vdash M : A \\
    N : A \\
    \IsTrue{\phi} \vdash M = N : A
  }{
    N : A \mid \phi \to M
  }
\end{mathparpagebreakable}

We internalize this judgment as the \emph{cubical subtype} connective.

\begin{mathparpagebreakable}
  \inferrule[sub formation]{
    \IsTp{A} \\
    \phi : \COF \\
    \IsTrue{\phi} \vdash M : A
  }{
    \IsTp{\TpSub{A}{\phi}{M}}
  }
  \and
  \inferrule[sub intro]{
    \Hint{\IsTp{A}} \\
    \Hint{\phi : \COF} \\
    \Hint{\IsTrue{\phi} \vdash M : A} \\
    N : A \mid \phi \to M
  }{
    \SubIn{N} : \TpSub{A}{\phi}{M}
  }
  \and
  \inferrule[sub elimination]{
    \Hint{\IsTp{A}} \\
    \Hint{\phi : \COF} \\
    \Hint{\IsTrue{\phi} \vdash M : A} \\
    N : \TpSub{A}{\phi}{M}
  }{
    \SubOut{N} : A \mid \phi \to M
  }
\end{mathparpagebreakable}

\section{The universe}

We consider a weak universe \`a la Tarski closed under dependent product and sum
and, for the sake of simplicity, \emph{itself}; this inconsistent universe would
be replaced by a hierarchy of universes.

\begin{mathparpagebreakable}
  \inferrule[univ formation]{
  }{
    \IsTp{\TpUniv}
  }
  \and
  \inferrule[el formation]{
    \Code{A} : \TpUniv
  }{
    \IsTp{\TpEl{\Code{A}}}
  }
  \and
  \inferrule[pi code]{
    \Code{A} : \TpUniv\\
    \Code{B} : \TpEl{\Code{A}}\to {\TpUniv}
  }{
    \CodeProd{x:\Code{A}}{\Code{B}\prn{x}} : \TpUniv
  }
  \and
  \inferrule[sigma code]{
    \Code{A} : \TpUniv\\
    \Code{B} : \TpEl{\Code{A}}\to {\TpUniv}
  }{
    \CodeSum{x:\Code{A}}{\Code{B}\prn{x}} : \TpUniv
  }
  \and
  \inferrule[universe code]{
  }{
    \CodeUniv : \TpUniv
  }
  \and
  \inferrule[pi decoding]{
    \Code{A} : \TpUniv\\
    \Code{B} : \TpEl{\Code{A}}\to {\TpUniv}
  }{
    \ElInOut : \TpEl{\CodeProd{x:\Code{A}}{\Code{B}\prn{x}}} \cong
    \Prod{x : \TpEl{\Code{A}}}{\TpEl{\Code{B}\prn{x}}}
  }
  \and
  \inferrule[sigma decoding]{
    \Code{A} : \TpUniv\\
    \Code{B} : \TpEl{\Code{A}}\to {\TpUniv}
  }{
    \ElInOut : \TpEl{\CodeSum{x:\Code{A}}{\Code{B}\prn{x}}} \cong
    \Sum{x : \TpEl{\Code{A}}}{\TpEl{\Code{B}\prn{x}}}
  }
  \and
  \inferrule[path decoding]{
    \Code{A} : \DIM \to \TpUniv\\
    M : \TpEl{\Code{A}\prn{0}}\\
    N : \TpEl{\Code{A}\prn{1}}
  }{
    \ElInOut : \TpEl{\CodePath{\Code{A}}{M}{N}} \cong
    \Prod{i : \DIM}{
      \TpSub{\TpEl{\Code{A}\prn{i}}}{\Boundary{i}}{
        \brk{i=0 \to M \delimmid i=1 \to N}
      }
    }
  }
  \and
  \inferrule[universe code decoding]{
  }{
    \ElInOut : \TpEl{\CodeUniv} \cong \TpUniv
  }
\end{mathparpagebreakable}

Kan operations are a structure on codes of types.

\begin{mathparpagebreakable}
  \inferrule[hcom structure]{
    \Code{A} : \TpUniv\\
    r,s : \DIM\\
    \phi : \COF\\
    M : \Prod{i:\DIM}{\TpPrf{i=r\lor\phi}\to\TpEl{\Code{A}}}
  }{
    \HCom{\Code{A}}{r}{s}{\phi}{M} : \TpEl{\Code{A}}
    \mid
    r=s\lor\phi \to M\prn{s,*}
  }
  \and
  \inferrule[coe structure]{
    \Code{A} : \DIM\to \TpUniv\\
    r,s:\DIM\\
    M : \TpEl{\Code{A}\prn{r}}
  }{
    \Coe{\Code{A}}{r}{s}{M} : \TpEl{\Code{A}\prn{s}} \mid
    r=s\to M
  }
  \and
  \inferrule[com structure]{
    \Code{A} : \DIM\to \TpUniv\\
    r,s : \DIM\\
    \phi : \COF\\
    M : \Prod{i:\DIM}{\TpPrf{i=r\lor\phi}\to\TpEl{\Code{A}\prn{i}}}
  }{
    \Com{\Code{A}}{r}{s}{\phi}{M} : \TpEl{\Code{A}\prn{s}}
    \mid
    r=s\lor\phi \to M\prn{s,*}
  }
\end{mathparpagebreakable}

In the above, $\mathsf{com}$ is definitionally equal to its standard
decomposition into $\mathsf{hcom}$ and $\mathsf{coe}$.

\begin{cool}
We intend for the bureaucracy of $\ElInOut$ (and $\SubIn{-},\SubOut{-}$) to be
handled by \texttt{cooltt}'s elaborator. One might informally imagine $\Code{A}:\TpUniv$ as a pair of a type
$\TpEl{\Code{A}}$ and evidence that $\TpEl{\Code{A}}$ is Kan.
\end{cool}

\begin{cool}
  %
  Prior implementations\footnote{Namely \texttt{cubicaltt}, \RedPRL{}, Cubical Agda, and \redtt{}\citep{cchm:cubicaltt,
  vezzosi-mortberg-abel:2019,redtt:2018,redprl:2018}.} have
  inherited the artificial separation of the ``cap'' and the ``tube'' parts of
  the composition data, inspired by the cubical type theoretic
  literature~\citep{cchm:2017,angiuli-favonia-harper:2017,orton-pitts:2016,abcfhl:2019}.
  While this is unnecessary and inelegant in general, the situation is
  particularly worsened in the context of Agda where the comparatively
  under-developed judgmental support for cubical subtypes leads to a
  discrepency between the types of arguments passed to \textsf{hcomp}
  (primitive) and \textsf{hfill} (derived).
%
  The main design challenge is to express the fact that the cap \emph{fits}
  onto the tube; one can use cubical subtypes to impose this coherence
  condition, but cubical subtypes are clumsy without good judgmental support
  during elaboration.

  Consequently, Agda over-approximates the type of the primitive operator
  \textsf{hcomp} and then employs an ad hoc check for the coherence condition.
  However, the \emph{derived} function \textsf{hfill} does not enjoy the
  special treatment of \textsf{hcomp}, and therefore must use cubical subtypes. In other
  words, the operators are either semantically correct but clumsy
  (\textsf{hfill}) or convenient but semantically broken (\textsf{hcomp}).
  There are at least two ways to fix this:
  \begin{enumerate}
    \item Consolidate ``cap'' and ``tube'' into one system.
    \item Good judgmental support for cubical subtypes.
  \end{enumerate}

  We have done both in \texttt{cooltt}.
\end{cool}

\subsection{Rules for composite types}

We proceed by equipping the universe with composite types and V types, which
ensure it is Kan and univalent, respectively. The homogeneous composition
structure on the universe corresponds to a code for composite types; we
characterize the elements of this code \emph{directly} rather than via some
isomorphism (this is simplest in the case of cubically unstable type codes).

\begin{mathparpagebreakable}
  \inferrule[formal hcom code]{
    r,s : \DIM\\
    \phi : \COF\\
    \Code{A} : \Prod{i:\DIM}{\TpPrf{i=r\lor\phi}\to\TpUniv}
  }{
    \CodeHCom{r}{s}{\phi}{\Code{A}} : \TpUniv
    \mid
    r=s\lor\phi
    \to
    \Code{A}\prn{s,*}
  }
  \and
  \inferrule[hcom-univ computation]{
    \Hint{\Code{A} : \Prod{i:\DIM}{\TpPrf{i=r\lor\phi}\to\TpEl{\CodeUniv}}}
  }{
    \HCom{\CodeUniv}{r}{s}{\phi}{\Code{A}} =
    \ElIn{
      \CodeHCom{r}{s}{\phi}{
        \lambda i,*. \ElOut{\Code{A}\prn{i,*}}
      }
    }
    :
    \TpEl{\CodeUniv}
  }
  \and
  \inferrule[composite type introduction]{
    \Hint{r,s:\DIM}\\
    \Hint{\phi:\COF}\\
    \Hint{
      \Code{A} : \Prod{i:\DIM}{
        \TpPrf{i=r\lor\phi}\to\TpUniv
      }
    }\\
    N : \Prod{*:\TpPrf{\phi}}{\TpEl{\Code{A}\prn{s,*}}}\\
    M :
    \TpEl{\Code{A}\prn{r,*}} \mid
    \phi \to
    \Coe{\lambda i.\Code{A}\prn{i,*}}{s}{r}{
      N\prn{*}
    }
  }{
    \TmBox{r}{s}{\phi}{N}{M} : \TpEl{\CodeHCom{r}{s}{\phi}{\Code{A}}}
    \mid
    r=s \to M\mid
    \phi \to N\prn{*}
  }
  \and
  \inferrule[composite type elimination]{
    \Hint{r,s:\DIM}\\
    \Hint{\phi:\COF}\\
    \Code{A} : \Prod{i:\DIM}{
      \TpPrf{i=r\lor\phi}\to\TpUniv
    }
    \\
    M : \TpEl{\CodeHCom{r}{s}{\phi}{\Code{A}}}
  }{
    \TmCap{r}{s}{\phi}{\Code{A}}{M} :
    \TpEl{\Code{A}\prn{r,*}}
    \mid
    r=s\to M
    \mid
    \phi\to\Coe{\lambda i.\Code{A}\prn{i,*}}{s}{r}{M}
  }
  \and
  \inferrule[composite type computation]{
    \Hint{r,s:\DIM}\\
    \Hint{\phi:\COF}\\
    \Hint{
      \Code{A} : \Prod{i:\DIM}{
        \TpPrf{i=r\lor\phi}\to\TpUniv
      }
    }\\
    \Hint{N : \Prod{*:\TpPrf{\phi}}{\TpEl{\Code{A}\prn{s,*}}}}\\
    \Hint{
      M :
      \TpEl{\Code{A}\prn{r,*}}
      \mid
      \phi\to
      \Coe{\lambda i.\Code{A}\prn{i,*}}{s}{r}{
        N\prn{*}
      }
    }
  }{
    \TmCap{r}{s}{\phi}{\Code{A}}{
      \TmBox{r}{s}{\phi}{N}{M}
    }
    =
    M
    : \TpEl{\Code{A}\prn{r,*}}
  }
  \and
  \inferrule[composite type uniqueness]{
    \Hint{r,s:\DIM}\\
    \Hint{\phi:\COF}\\
    \Hint{
      \Code{A} : \Prod{i:\DIM}{
        \TpPrf{i=r\lor\phi}\to\TpUniv
      }
    }\\
    \Hint{
      M : \TpEl{\CodeHCom{r}{s}{\phi}{\Code{A}}}
    }
  }{
    M =
    \TmBox{r}{s}{\phi}{
      \lambda{*}. M
    }{
      \TmCap{r}{s}{\phi}{\Code{A}}{M}
    }
    : \TpEl{\CodeHCom{r}{s}{\phi}{\Code{A}}}
  }
\end{mathparpagebreakable}

\begin{warning}
  %
  The arguments to $\mathsf{box}$ go in the order forced by their typing
  constraints, but please note that this is the opposite of what appeared in
  \citep{angiuli:2019}.
  %
\end{warning}



\subsection{Rules for V types}

We begin by defining some definitional extensions for equivalences.
\begin{gather*}
  \begin{array}{l}
    \CodeIsContr : \TpUniv\to\TpUniv\\
    \CodeIsContr{\Code{C}} \coloneq \CodeSum{x:\Code{C}}{\CodeProd{y:\Code{C}}{\CodePath{\lambda\Dummy.\Code{C}}{x}{y}}}
  \end{array}
  \\
  \begin{array}{l}
    \CodeFiber : \Prod{\Code{A}:\TpUniv}{\Prod{\Code{B}:\TpUniv}{\prn{\TpEl{\Code{A}}\to\TpEl{\Code{B}}}\to\TpEl{\Code{B}}\to\TpUniv}}\\
    \CodeFiber{\Code{A}}{\Code{B}}{f}{y} \coloneq \CodeSum{x:\Code{A}}{\CodePath{\lambda\Dummy.\Code{B}}{f\prn{x}}{y}}
  \end{array}
  \\
  \begin{array}{l}
    \CodeEquiv : \TpUniv\to\TpUniv\to\TpUniv\\
    \CodeEquiv{\Code{A}}{\Code{B}} \coloneq
    \CodeSum{
      f:\Code{A}\CodeTo\Code{B}
    }{
      \CodeProd{y:\Code{B}}{
        \CodeIsContr{\CodeFiber{\Code{A}}{\Code{B}}{\ElOut{f}}{y}}
      }
    }
  \end{array}
  \\
  \begin{array}{l}
    \TmEquivApp : \Prod{\Code{A}:\TpUniv}{\Prod{\Code{B}:\TpUniv}{\TpEl{\CodeEquiv{\Code{A}}{\Code{B}}}\to\TpEl{\Code{A}}\to\TpEl{\Code{B}}}}\\
    \TmEquivApp{\Code{A}}{\Code{B}}{E} \coloneq \ElOut{\TmFst{\ElOut{E}}}
  \end{array}
\end{gather*}

We may now specify the rules for the V types.
\begin{mathparpagebreakable}
  \inferrule[v type code]{
    r : \DIM\\
    \Code{A} : \TpPrf{r=0}\to\TpUniv\\
    \Code{B} : \TpUniv\\
    E : \Prod{*:\TpPrf{r=0}}{\TpEl{\CodeEquiv{\Code{A}\prn{*}}{\Code{B}}}}
  }{
    \CodeV{r}{\Code{A}}{\Code{B}}{E} : \TpUniv
    \mid
    r=0
    \to
    \Code{A}\prn{*}
    \mid
    r=1
    \to
    \Code{B}
  }
  \\
  \inferrule[v type introduction]{
    \Hint{r:\DIM}\\
    \Hint{\Code{A} : \TpPrf{r=0}\to\TpUniv}\\
    \Hint{\Code{B} : \TpUniv}\\
    \Hint{E : \Prod{*:\TpPrf{r=0}}{\TpEl{\CodeEquiv{\Code{A}\prn{*}}{\Code{B}}}}}\\
    \\\\
    M : \Prod{*:\TpPrf{r=0}}{\TpEl{\Code{A}\prn{*}}}\\
    N : \TpEl{\Code{B}} \mid r=0 \to \TmEquivApp{\Code{A}\prn{*}}{\Code{B}}{E\prn{*}}{M\prn{*}}
  }{
    \TmVin{r}{E}{M}{N} : \TpEl{\CodeV{r}{\Code{A}}{\Code{B}}{E}}
    \mid r=0 \to M\prn{*}
    \mid r=1 \to N
  }
  \and
  \inferrule[v type elimination]{
    \Hint{r:\DIM}\\
    \Hint{\Code{A} : \TpPrf{r=0}\to\TpUniv}\\
    \Hint{\Code{B} : \TpUniv}\\
    \Hint{E : \Prod{*:\TpPrf{r=0}}{\TpEl{\CodeEquiv{\Code{A}\prn{*}}{\Code{B}}}}}\\
    M : \TpEl{\CodeV{r}{\Code{A}}{\Code{B}}{E}}
  }{
    \TmVproj{r}{\Code{A}}{\Code{B}}{E}{M} : \TpEl{\Code{B}}
    \mid r=0 \to \TmEquivApp{\Code{A}\prn{*}}{\Code{B}}{E\prn{*}}{M}
    \mid r=1 \to M
  }
  \and
  \inferrule[v type computation]{
    \Hint{r:\DIM}\\
    \Hint{\Code{A} : \TpPrf{r=0}\to\TpUniv}\\
    \Hint{\Code{B} : \TpUniv}\\
    \Hint{E : \Prod{*:\TpPrf{r=0}}{\TpEl{\CodeEquiv{\Code{A}\prn{*}}{\Code{B}}}}}\\
    \\\\
    \Hint{M : \Prod{*:\TpPrf{r=0}}{\TpEl{\Code{A}\prn{*}}}}\\
    \Hint{N : \TpEl{\Code{B}} \mid r=0 \to \TmEquivApp{\Code{A}\prn{*}}{\Code{B}}{E\prn{*}}{M\prn{*}}}
  }{
    \TmVproj{r}{\Code{A}}{\Code{B}}{E}{\TmVin{r}{E}{M}{N}} = N : \TpEl{\Code{B}}
  }
  \and
  \inferrule[v type uniqueness]{
    \Hint{r:\DIM}\\
    \Hint{\Code{A} : \TpPrf{r=0}\to\TpUniv}\\
    \Hint{\Code{B} : \TpUniv}\\
    \Hint{E : \Prod{*:\TpPrf{r=0}}{\TpEl{\CodeEquiv{\Code{A}\prn{*}}{\Code{B}}}}}\\
    \Hint{M : \TpEl{\CodeV{r}{\Code{A}}{\Code{B}}{E}}}
  }{
    M = \TmVin{r}{E}{\lambda{*}.M}{\TmVproj{r}{\Code{A}}{\Code{B}}{E}{M}} : \TpEl{\CodeV{r}{\Code{A}}{\Code{B}}{E}}
  }
  \and
\end{mathparpagebreakable}





\begin{landscape}

\section{Composition and coercion algorithms}

\subsection{Composition in V types}

\[
  \inferrule{
    \Hint{s:\DIM}\\
    \Hint{\Code{A} : \TpPrf{s=0}\to\TpUniv}\\
    \Hint{\Code{B} : \TpUniv}\\
    \Hint{E : \Prod{*:\TpPrf{s=0}}{\TpEl{\CodeEquiv{\Code{A}\prn{*}}{\Code{B}}}}}\\
    \\\\
    \Hint{r,r':\DIM}\\
    \Hint{
      M : \Prod{i:\DIM}{
        \TpPrf{i=r\lor\phi}\to\TpEl{
          \CodeV{s}{\Code{A}}{\Code{B}}{E}
        }
      }
    }
    \\\\\\
    {
      \begin{array}{l}
        \tilde{O} : \Prod{\Code{X}:\TpUniv}{\prn{\Prod{i:\DIM}{\TpPrf{i=r\lor\phi}\to\TpEl{\Code{X}}}}\to\DIM\to\TpEl{\Code{X}}}\\
        \tilde{O}\prn{\Code{X},N,i} \coloneq \HCom{\Code{X}}{r}{i}{\phi}{N}
      \end{array}
    }
    \\\\\\
    {
      \begin{array}{l}
        \tilde{P} : \Prod{i:\DIM}{\TpPrf{i=r\lor\phi\lor\Boundary{s}}\to\TpEl{\Code{B}}}\\
        \tilde{P}\prn{i,*} \coloneq \brk{
          i=r\lor\phi \to \TmVproj{s}{\Code{A}}{\Code{B}}{E}{M\prn{i,*}} \delimmid
          s=0 \to \TmEquivApp{\Code{A}\prn{*}}{\Code{B}}{E\prn{*}}{\tilde{O}\prn{\Code{A}\prn{*},M,i}} \delimmid
          s=1 \to \tilde{O}\prn{\Code{B},M,i}
        }
      \end{array}
    }
  }{
    \HCom{
      \CodeV{s}{\Code{A}}{\Code{B}}{E}
    }{r}{r'}{\phi}{M}
    =
    \TmVin{s}{E}{\lambda{*}.\tilde{O}\prn{\Code{A}\prn{*},M,r'}}{\HCom{\Code{B}}{r}{r'}{\phi\lor\Boundary{s}}{\tilde{P}}}
    :
    \TpEl{
      \CodeV{s}{\Code{A}}{\Code{B}}{E}
    }
  }
\]


\subsection{Composition in composite types}
\[
  \inferrule{
    \Hint{s,s' : \DIM}\\
    \Hint{\psi : \COF}\\
    \Hint{\Code{A} : \Prod{j:\DIM}{\TpPrf{j=s\lor\psi}\to\TpUniv}}\\\\
    \Hint{r,r' : \DIM}\\
    \Hint{\phi : \COF}\\
    \Hint{
      M : \Prod{i:\DIM}{
        \Prod{*:\TpPrf{i=r\lor\phi}}{
          \TpEl{
            \CodeHCom{s}{s'}{\psi}{\Code{A}}
          }
        }
      }
    }
    \\\\\\
    {
      \begin{array}{l}
        \tilde{O} : \Prod{i:\DIM}{\Prod{*:\TpPrf{i=r\lor \phi}}{\TpEl{\Code{A}\prn{s,*}}}}\\
        \tilde{O}\prn{i,*} \coloneq \TmCap{s}{s'}{\psi}{\Code{A}}{M\prn{i,*}}
      \end{array}
    }
    \and
    {
      \begin{array}{l}
        \tilde{P} : \Prod{i:\DIM}{\Prod{*:\TpPrf{s=s'\lor \psi}}{\TpEl{\Code{A}\prn{s',*}}}}\\
        \tilde{P}\prn{i,*} = \HCom{\Code{A}\prn{s',*}}{r}{i}{\phi}{M}
      \end{array}
    }
    \\\\\\
    {
      \begin{array}{l}
        \tilde{T} : \Prod{i:\DIM}{\Prod{*:\TpPrf{i=r\lor\phi\lor\psi\lor{s=s'}}}{\TpEl{\Code{A}\prn{s,*}}}}\\
        \tilde{T}\prn{i,*} \coloneq
        \brk{
          i=r\lor\phi \to \tilde{O}\prn{i,*}
          \delimmid
          \psi \to \Coe{\lambda j. \Code{A}\prn{j,*}}{s'}{s}{\tilde{P}\prn{i,*}}
          \delimmid
          s=s' \to \tilde{P}\prn{i,*}
        }
      \end{array}
    }
  }{
    \HCom{
      \CodeHCom{s}{s'}{\psi}{\Code{A}}
    }{r}{r'}{\phi}{M}
    =
    \TmBox{s}{s'}{\psi}{\lambda{*}.\tilde{P}\prn{r',*}}{
      \HCom{\Code{A}\prn{s,*}}{r}{r'}{\phi\lor\psi\lor{s=s'}}{\tilde{T}}
    }
    :
    \TpEl{
      \CodeHCom{s}{s'}{\psi}{\Code{A}}
    }
  }
\]


\subsection{Coercion in V types}
\NewDocumentCommand{\CodeFiberAlt}{g}{\widehat{\mathsf{Fiber'}}\IfValueT{#1}{\prn{#1}}}
\NewDocumentCommand{\TooLong}{}{\textcolor{RedDevil}{\cdots\text{\textit{too long}}\cdots}}

\[
  \inferrule{
    \Hint{s_\bullet : \DIM\to\DIM}\\
    \Hint{\Code{A}_\bullet : \Prod{i:\DIM}{\TpPrf{s_i=0}\to\TpUniv}}\\
    \Hint{\Code{B}_\bullet : \DIM\to\TpUniv}\\
    \Hint{E_\bullet : \Prod{i:\DIM}{\Prod{*:\TpPrf{s_i=0}}{\TpEl{\CodeEquiv{\Code{A}_i\prn{*}}{\Code{B}_i}}}}}\\
    \\\\
    \Hint{r,r':\DIM}\\
    \Hint{M : \TpEl{\CodeV{s_r}{\Code{A}_r}{\Code{B}_r}{E_r}}}
    \\\\\\
    {
      \begin{array}{l}
        \tilde{F} : \Prod{i:\DIM}{\TpPrf{s_i=0}\to\TpEl{\Code{A}_i\prn{*}}\to\TpEl{\Code{B}_i}}
        \\
        \tilde{F}\prn{i,*} \coloneq \TmEquivApp{\Code{A}_{i}\prn{*}}{\Code{B}_{i}}{E_{i}\prn{*}}
      \end{array}
    }
    \and
    {
      \begin{array}{l}
        \tilde{O} : \TpEl{\Code{B}_{r'}}\\
        \tilde{O} \coloneq \Coe{\Code{B}_\bullet}{r}{r'}{\TmVproj{s_r}{\Code{A}_r}{\Code{B}_r}{E_r}{M}}
      \end{array}
    }
    \\\\\\
    {
      \begin{array}{l}
        \CodeFiberAlt : \TpPrf{s_{r'}=0}\to\TpUniv
        \\
        \CodeFiberAlt{*} \coloneq \CodeFiber{\Code{A}_{r'}\prn{*}}{\Code{B}_{r'}}{\tilde{F}\prn{r',*}}{\tilde{O}}
      \end{array}
    }
    \and
    {
      \begin{array}{l}
        \tilde{I} :
        \Prod{*:\TpPrf{s_{r'}=0}}{
          \TpEl{\CodeIsContr{\CodeFiberAlt{*}}}
        }
        \\
        \tilde{I}\prn{*} \coloneq \ElOut{\TmSnd{\ElOut{E_{r'}\prn{*}}}}\prn{\tilde{O}}
      \end{array}
    }
    \\\\\\
    {
      \begin{array}{l}
        \tilde{P} :
        \Prod{*:\TpPrf{s_{r'}=0}}{
          \TpEl{\CodeFiberAlt{*}}
        }
        \\
        \tilde{P}\prn{*} \coloneq \TmFst{\ElOut{\tilde{I}\prn{*}}}
      \end{array}
    }
    \and
    {
      \begin{array}{l}
        \tilde{Q} :
        \Prod{*:\TpPrf{s_{r'}=0}}{
          \Prod{
            p:\TpEl{\CodeFiberAlt{*}}
          }{
            \TpEl{
              \CodePath{
                \lambda\Dummy.\CodeFiberAlt{*}
              }{
                \tilde{P}\prn{*}
              }{
                p
              }
            }
          }
        }
        \\
        \tilde{Q}\prn{*} \coloneq
        \ElOut{
          \TmSnd{
            \ElOut{
              \tilde{I}\prn{*}
            }
          }
        }
      \end{array}
    }
    \\\\\\
    {
      \begin{array}{l}
        \tilde{R} :
        \Prod{*:\TpPrf{\forall k.s_k=0}}{
          \TpEl{
            \CodePath{\lambda\Dummy.\Code{B}_{r'}}{
              \tilde{F}\prn{r',*,\Coe{\Code{A}_\bullet\prn{*}}{r}{r'}{M}}
            }{
              \tilde{O}
            }
          }
        }
        \\
        \tilde{R}\prn{*} \coloneq
        \Coe{
          \lambda i.
          \CodePath{
            \lambda\Dummy.\Code{B}_i
          }{
            \tilde{F}\prn{i,*,\Coe{\Code{A}_\bullet\prn{*}}{r}{i}{M}}
          }{
            \Coe{\Code{B}_\bullet}{r}{i}{\tilde{F}\prn{r,*,M}}
          }
        }{r}{r'}{
          \ElIn{\lambda\Dummy.\SubIn{\tilde{F}\prn{r,*,M}}}
        }
        % alternate definition of \tilde{R}:
        %\\
        %\tilde{R}\prn{*} \coloneq
        %\ElIn{
        %  \lambda i.
        %  \SubIn{
        %    \Com{\Code{B}_\bullet}{r}{r'}{\Boundary{i}}{
        %      \lambda j,*.
        %      \brk{
        %        j=r\lor i=0 \to
        %        \tilde{F}\prn{j,*,\Coe{\Code{A}_\bullet\prn{*}}{r}{j}{M}}
        %        \delimmid
        %        i=1\to
        %        \Coe{\Code{B}_\bullet}{r}{j}{\tilde{F}\prn{r,*,M}}
        %      }
        %    }
        %  }
        %}
      \end{array}
    }
    \\\\\\
    {
      \begin{array}{l}
        \tilde{S} :
        \Prod{*:\TpPrf{s_{r'}=0\land\prn{\prn{\forall k.s_k=0}\lor r=r'}}}{
          \TpEl{
            \CodePath{
              \lambda\Dummy.\CodeFiberAlt{*}
            }{
              \tilde{P}\prn{*}
            }{
              \ElIn*{
                \brk{
                  \prn{\forall k.s_k = 0}\to\TooLong
                  \delimmid
                  r=r'\to\TooLong
                }
              }
            }
          }
        }
        \\
        \tilde{S}\prn{*} \coloneq
        \tilde{Q}\prn{
          *,
          \ElIn*{
            \brk{
              \prn{\forall k.s_k = 0}\to\TmPair{\Coe{\Code{A}_\bullet\prn{*}}{r}{r'}{M}}{\tilde{R}\prn{*}}
              \delimmid
              r=r'\to \TmPair{M}{\ElIn{\lambda\Dummy.\SubIn{\TmVproj{s_r}{\Code{A}_r}{\Code{B}_r}{E_r}{M}}}}
            }
          }
        }
      \end{array}
    }
    \\\\\\
    {
      \begin{array}{l}
        \tilde{T} :
        \Prod{*:\TpPrf{s_{r'}=0}}{
          \TpEl{\CodeFiberAlt{*}}
        }
        \\
        \tilde{T}\prn{*} \coloneq
        \HCom{
          \CodeFiberAlt{*}
        }{0}{1}{\parens{\forall k.s_k = 0}\lor r=r'}{
          \lambda j,*.
          \brk{
            j=0 \to \tilde{P}\prn{*}
            \delimmid
            \prn{\forall k.s_k = 0}\lor r=r'\to \SubOut{\ElOut{\tilde{S}\prn{*}}\prn{j}}
          }
        }
      \end{array}
    }
    \\\\\\
    {
      \begin{array}{l}
        \tilde{U} : \TpEl{\Code{B}_{r'}}\\
        \tilde{U} \coloneq \HCom{\Code{B}_{r'}}{1}{s_{r'}}{r=r'\lor{s_{r'}=0}}{
          \lambda k,*.
          \brk{
            k=1\lor r=r'\to \tilde{O}
            \delimmid
            s_{r'}=0 \to \SubOut{\ElOut{\TmSnd{\ElOut{\tilde{T}\prn{*}}}}\prn{k}}
          }
        }
      \end{array}
    }
  }{
    \Coe{
      \CodeV{s_\bullet}{\Code{A}_\bullet}{\Code{B}_\bullet}{E_\bullet}
    }{r}{r'}{M}
    =
    \TmVin{s_{r'}}{E_{r'}}{\lambda{*}.\TmFst{\ElOut{\tilde{T}\prn{*}}}}{\tilde{U}}
    :
    \TpEl{\CodeV{s_{r'}}{\Code{A}_{r'}}{\Code{B}_{r'}}{E_{r'}}}
  }
\]



\subsection{Coercion in composite types}

\[
  \inferrule{
    \Hint{s_\bullet, s'_\bullet : \DIM\to\DIM}\\
    \Hint{\phi_\bullet : \DIM\to\COF}\\
    \Hint{
      \Code{A}_\bullet :
      \Prod{i:\DIM}{
        \Prod{j:\DIM}{
          \Prod{*:\TpPrf{j=s_{i}\lor{\phi_{i}}}}{
            \TpUniv
          }
        }
      }
    }
    \\\\
    \Hint{r,r' : \DIM}\\
    \Hint{
      M : \TpEl{
        \CodeHCom{s_{r}}{s'_{r}}{\phi_{r}}{\Code{A}_{r}}
      }
    }
    \\\\\\
    {
      \begin{array}{l}
        \tilde{N} : \Prod{i,j:\DIM}{
          \Prod{*:\TpPrf{\forall i.\phi_i}}{
            \TpEl{\Code{A}_{i}\prn{j,*}}
          }
        }\\
        \tilde{N}\prn{i,j,*} \coloneq
        \Coe{\Code{A}_i\prn{\bullet,*}}{s'_i}{j}{\Coe{\Code{A}_\bullet\prn{s'_\bullet,*}}{r}{i}{M}}
      \end{array}
    }
    \\\\\\
    {
      \begin{array}{l}
        \tilde{O} : \DIM \to \TpEl{\Code{A}_{r}\prn{s_r,*}}
        \\
        \tilde{O}\prn{j} \coloneq
        \HCom{\Code{A}_{r}\prn{s_r,*}}{s'_{r}}{j}{\phi_{r}}{
          \lambda k,*.
          \brk{
            k=s'_r \to \TmCap{s_r}{s'_r}{\phi_r}{\Code{A}_r}{M}
            \delimmid
            \phi_r \to
            \Coe{\Code{A}_r\prn{\bullet,*}}{k}{s_r}{
              \Coe{\Code{A}_r\prn{\bullet,*}}{s'_r}{k}{M}
            }
          }
        }
      \end{array}
    }
    \\\\\\
    {
      \begin{array}{l}
        \tilde{P} : \TpEl{\Code{A}_{r'}\prn{s_{r'},*}} \\
        \tilde{P} \coloneq
        \Com{\Code{A}_\bullet\prn{s_\bullet,*}}{r}{r'}{\forall i.\phi_i \lor \forall i.\prn{s_i=s'_i}}{
          \lambda i,*.
          \brk{
            i=r \to \tilde{O}\prn{s_r}
            \delimmid
            \forall i.\phi_i \to \tilde{N}\prn{i,s_i,*}
            \delimmid
            \forall i.(s_i=s'_i) \to \Coe{\Code{A}_\bullet\prn{s_\bullet,*}}{r}{i}{M}
          }
        }
      \end{array}
    }
    \\\\\\
    {
      \begin{array}{l}
        \tilde{Q} : \Prod{j:\DIM}{
          \Prod{*:\TpPrf{\phi_{r'}}}{
            \TpEl{\Code{A}_{r'}\prn{j,*}}
          }
        }\\
        \tilde{Q}\prn{j,*} \coloneq
        \Com{\Code{A}_{r'}\prn{\bullet,*}}{s_{r'}}{j}{r=r'\lor\forall i.\phi_i}{
          \lambda k,*.\brk{
            k=s_{r'} \to \tilde{P}
            \delimmid
            r=r' \to \Coe{\Code{A}_{r'}\prn{\bullet,*}}{s'_{r'}}{k}{M}
            \delimmid
            \forall i.\phi_i \to \tilde{N}\prn{r',k,*}
          }
        }
      \end{array}
    }
    \\\\\\
    {
      \begin{array}{l}
        \tilde{H} : \TpEl{\Code{A}_{r'}\prn{s_{r'},*}} \\
        \tilde{H} \coloneq
        \HCom{\Code{A}_{r'}(s_{r'},*)}{s_{r'}}{s'_{r'}}{\phi_{r'} \lor r=r'}{
          \lambda j,*.\brk{
            j=s_{r'}\to \tilde{P}
            \delimmid
            \phi_{r'} \to \Coe{\Code{A}_{r'}\prn{\bullet,*}}{j}{s_{r'}}{\tilde{Q}\prn{j,*}}
            \delimmid
            r=r' \to \tilde{O}\prn{j}
          }
        }
      \end{array}
    }
  }{
    \Coe{
      \lambda i.
      \CodeHCom{s_{i}}{s'_{i}}{\phi_{i}}{\Code{A}_i}
    }{r}{r'}{M}
    =
    \TmBox{s_{r'}}{s'_{r'}}{\phi_{r'}}{
      \lambda{*}.{\tilde{Q}\prn{s'_{r'},*}}
    }{
      \tilde{H}
    }
    :
    \TpEl{
      \CodeHCom{s_{r'}}{s'_{r'}}{\phi_{r'}}{\Code{A}_{r'}}
    }
  }
\]
\end{landscape}

\clearpage
\nocite{*}
\printbibliography

\end{document}
